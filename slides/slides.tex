\documentclass[10pt, xetex, hyperref={pdfpagelabels=false}]{beamer}

% --------------------------------------------------------------------------
% Packages
% --------------------------------------------------------------------------
\usepackage[english]{babel}
\usepackage{amsmath, amsthm, amssymb}
\usepackage{bussproofs}
\EnableBpAbbreviations
\def\extraVskip{3pt}

\usepackage{fancyvrb}
\usepackage{geometry}
\usepackage{graphicx}
\usepackage{lmodern}

\usepackage{url}

% --------------------------------------------------------------------------
% Tikz Configuration
% --------------------------------------------------------------------------
\usepackage{tikz}
\usetikzlibrary{positioning}
\usetikzlibrary{calc}
\usepackage{rotating}

% --------------------------------------------------------------------------
% My palette.
% --------------------------------------------------------------------------
\definecolor{aliceblue}{rgb}{0.94, 0.97, 1.0}
\definecolor{energy}{RGB}{49,247,250}
\definecolor{delicate}{RGB}{67,179,223}
\definecolor{faded}{RGB}{76,117,195}
\definecolor{blu}{RGB}{1,0,102}
\definecolor{plum}{RGB}{87,78,164}
\definecolor{petunias}{RGB}{109,80,139}
\definecolor{letour}{RGB}{101,41,105}
% --------------------------------------------------------------------------

% --------------------------------------------------------------------------
% Beamer configuration.
% --------------------------------------------------------------------------
\usetheme{default}
\usecolortheme{default}
\usefonttheme{serif}

\beamertemplatenavigationsymbolsempty
\setbeamertemplate{navigation symbols}{}
\hypersetup{pdfpagemode=UseNone}

% footer.
\setbeamercolor{headFoot}{fg=white, bg=blu}
\setbeamertemplate{footline}{
  \leavevmode%
  \hbox{%
  \begin{beamercolorbox}
    [wd=.8\paperwidth,ht=2.3ex,dp=1ex,left]{headFoot}%
    \hspace*{2ex}\textbf\insertshorttitle\hspace*{2mm}|
    \hspace*{2mm}\textbf\insertshortauthor
  \end{beamercolorbox}%
  \begin{beamercolorbox}
    [wd=.2\paperwidth,ht=2.3ex,dp=1ex,right]{headFoot}%
    \insertframenumber{}/\inserttotalframenumber\hspace*{2ex}
  \end{beamercolorbox}}%
  \vskip 0pt%
}

\setbeamerfont{frametitle}{size=\small,series=\bfseries}
\setbeamercolor{frametitle}{fg=white,bg=blu}

\setbeamerfont{framesubtitle}{size=\normalfont\scriptsize}
\setbeamercolor{framesubtitle}{fg=white, bg=blu}

\setbeamercolor{background canvas}{bg=white}
\setbeamercolor{normal text}{fg=black}

% \setbeamercolor{institute}{fg=blu}
\setbeamercolor{title}{fg=blu}
% \setbeamercolor{subtitle}{fg=blu}

% \setbeamercolor{titlelike}{fg=blu}
\setbeamerfont{footnote}{size=\tiny}
\setbeamercolor{footnote}{fg=gray}
\setbeamercolor{block title}{bg=white,fg=blu}
% \setbeamercolor{block body}{bg=aliceblue}
\setbeamercolor{item}{fg=blu} % color of bullets
\setbeamercolor{subitem}{fg=blu}
% \setbeamercolor{itemize/enumerate subbody}{fg=blu}
% \setbeamertemplate{itemize subitem}{{\textendash}}
% \setbeamerfont{itemize/enumerate subbody}{size=\footnotesize}
% \setbeamerfont{itemize/enumerate subitem}{size=\footnotesize}

% --------------------------------------------------------------------------
% Fonts
% --------------------------------------------------------------------------
\usefonttheme{professionalfonts}
\usefonttheme{serif}
\usepackage{fontspec}
\usepackage{mathtools}
\usepackage{unicode-math}

\newfontfamily\djvu[ExternalLocation=fonts/
  , BoldFont=DejaVuSansMono-Bold.ttf
  , BoldItalicFont=DejaVuSansMono-BoldOblique.ttf
  , ItalicFont=DejaVuSansMono-Oblique.ttf
  ]{DejaVuSansMono.ttf}

\setmonofont[ExternalLocation=fonts/
, BoldFont=DejaVuSansMono-Bold.ttf
, BoldItalicFont=DejaVuSansMono-BoldOblique.ttf
, ItalicFont=DejaVuSansMono-Oblique.ttf
]{DejaVuSansMono.ttf}

\setmathfont[ExternalLocation=fonts/
  ]{DejaVuMathTeXGyre.ttf}
\newfontfamily\mathfont{fonts/DejaVuMathTeXGyre.ttf}

\setmainfont[ExternalLocation=fonts/
  , BoldFont=SourceSansPro-Semibold.otf
  , BoldItalicFont=SourceSansPro-SemiboldIt.otf
  , ItalicFont=SourceSansPro-It.otf
  ]{SourceSansPro-Regular.otf}

% \setmonofont[ExternalLocation=fonts/
%   , BoldFont=SourceCodePro-Semibold.ttf
%   , BoldItalicFont=SourceCodePro-SemiboldIt.ttf
%   , ItalicFont=SourceCodePro-It.ttf
%   ]{SourceCodePro-Regular.ttf}

\newfontfamily\sourcecode[ExternalLocation=fonts/
  , BoldFont=SourceCodePro-Semibold.ttf
  , BoldItalicFont=SourceCodePro-SemiboldIt.ttf
  , ItalicFont=SourceCodePro-It.ttf
  ]{SourceCodePro-Regular.ttf}

% --------------------------------------------------------------------------
% Agda Source code
% --------------------------------------------------------------------------

\usepackage{minted}
\setminted[cagda]{
  bgcolor   = aliceblue
, fontsize  = \footnotesize
, frame     = none
% , framerule = 0.4pt
% , framesep  = 0pt
, style     = cagda
}

% --------------------------------------------------------------------------
% References
% --------------------------------------------------------------------------

\usepackage[autostyle]{csquotes}
\usepackage[
    backend=biber
  , doi=false
  , eprint=false
  , isbn=false
  , natbib=true
  , sortlocale=en_US
  , style=authoryear-icomp
  , url=false
]{biblatex}
\addbibresource{ref.bib}
\renewcommand*{\nameyeardelim}{\addcomma\addspace}
\usepackage{silence}
\WarningFilter{biblatex}{Patching footnotes failed}
\WarningFilter{hyperref}{Token not allowed in a PDF string}

% --------------------------------------------------------------------------
% Title and Author
% --------------------------------------------------------------------------

\title[The Simply Typed $\lambda$-Calculus]{The Simply Typed $\lambda$-Calculus}

\subtitle{(In \texttt{Agda})}
\date{\footnotesize 1th June 2017}
\author[Jonathan Prieto-Cubides]{Jonathan Prieto-Cubides}
\institute{
Master in Applied Mathematics\\
Logic and Computation Group\\
Universidad EAFIT\\
Medell\'in, Colombia}
% --------------------------------------------------------------------------

\newsavebox\agdapragma

\begin{document}
\setcounter{page}{1}

\begin{frame}[plain]
\titlepage
  \begin{tikzpicture}[overlay, remember picture]
   \tikzset{shift={(current page.center)}}
    \node[xshift=0cm,yshift=-3.2cm] (eafit)
      {\includegraphics[width=0.2\textwidth]{figures/eafit}};
  \end{tikzpicture}
\end{frame}


\begin{frame}{Contents}
\tableofcontents
\end{frame}

\section{Lambda Calculus}
\begin{frame}[fragile]{$\lambda$-Calculus}
\begin{definition}
The set of $\lambda$-terms denoted by $\Lambda$ is built up
from a set of variables $V$ using application and (function) abstraction.
\begin{align*}
x\in V                  &\Rightarrow x\in \Lambda, \\
M\in \Lambda, x\in V    &\Rightarrow (\lambda x. M) \in \Lambda,\\
M, N\in \Lambda         &\Rightarrow (MN) \in \Lambda, \\
M, N\in \Lambda, x\in V &\Rightarrow ((\lambda x. M)\,N) \in \Lambda.
\end{align*}
\end{definition}
\pause
\begin{minted}{cagda}
Name : Set
Name = String

data Expr : Set where
  var : Name → Expr
  lam : Name → Expr → Expr
  _∙_ : Expr → Expr → Expr
\end{minted}

\end{frame}

\section{Typed Lambda Calculus}
\begin{frame}{$\lambda\rightarrow$-Curry System}
\begin{definition}

\begin{columns}
  \begin{column}{0.5\textwidth}
    \begin{prooftree}
    \AxiomC{$M : σ ⟶ τ$}
    \AxiomC{$N : σ$}
    \BinaryInfC{$MN : τ$}
    \end{prooftree}
  \end{column}
  \begin{column}{0.5\textwidth}
    \begin{prooftree}
    \AxiomC{$[x : \sigma]^1$}
    \UnaryInfC{$\vdots$}
    \UnaryInfC{$M : \tau$}
    \RightLabel{$\not\_1$}
    \UnaryInfC{$λx.M : σ → \tau$}
    \end{prooftree}
  \end{column}
\end{columns}

\end{definition}
\end{frame}

\section{Decibility of Type Assignment}
\begin{frame}{Decibility of Type Assignment~\citep{barendregt2013lambda}}
% \begin{equation*}
% Γ ⊢ M : τ
% \end{equation*}
\begin{tabular}{ll}
{\color{blu} \textbf{Problem}} & {\color{blu} \textbf{Question}} \\
{\color{blu} Type-checking}    & Given $M$ and $τ$, $Γ⊢ M : τ$?  \\
{\color{blu} Typability}       & Given $M$ does exists a $σ$, $Γ⊢ M : σ$? \\
{\color{blu} Inhabitation}     & Given $τ$, does exists an $M$ such that $Γ⊢ M : σ$?\\
\end{tabular}

\begin{theorem}
\begin{itemize}
\item It is decidable whether a term is typable in $\lambda\rightarrow$.
\item If a term $M$ is typable in $\lambda\rightarrow$, then M has a principal type scheme, i.e.
a type $σ$ such that every possible type for $M$ is a subsitution instance of $σ$.
Moreover $σ$ is computable from $M$.
\end{itemize}
\end{theorem}

\begin{theorem}
Type checking for $\lambda\rightarrow$ is decidable.
\end{theorem}

\end{frame}


\section{Syntax}
\begin{frame}[fragile]{Syntax Based on ~\citep{cactus}, and \citep{nad}}
\vfill
\begin{minted}{cagda}
module Typing (U : Set) where

data Type : Set where
  base : U    → Type
  _⟶_  : Type → Type → Type
\end{minted}

\vfill
{\color{plum!45!black}\rule{\textwidth}{0.5pt}}
\vfill

\begin{minted}{cagda}
module Syntax (Type : Set) where

open import Data.String

Name : Set
Name = String

-- Type Judgments (x ∶ A).
data Formal : Set where
  _∶_ : Name → Type → Formal

data Expr : Set where
  var : Name   → Expr
  lam : Formal → Expr → Expr
  _∙_ : Expr   → Expr → Expr
\end{minted}
\vfill
\end{frame}

\begin{frame}[fragile]{\texttt{module Examples (Type : Set) where}}
\begin{minted}{cagda}
open import Syntax Type

postulate A : Type

x = var "x"
y = var "y"
z = var "z"

-- I, K, S : Expr

I = lam ("x" ∶ A) x                  -- λx.x, x ∈ A
K = lam ("x" ∶ A) (lam ("y" ∶ A) x)  -- λxy.x, x,y ∈ A
S =
  lam ("x" ∶ A)
    (lam ("y" ∶ A)
      (lam ("z" ∶ A)
        ((x ∙ z) ∙ (y ∙ z))))        -- λxyz.xz(yz), x,y,z ∈ A
\end{minted}
\end{frame}

\section{Well-Scoped Expressions}
\begin{frame}[fragile]{module Bound (Type : Set) where}
% Well-scoped Expresions with respect to a variable Binder.
\begin{minted}{cagda}
data Expr (n : ℕ) : Set where
  var : Fin n  → Expr n
  lam : Type   → Expr (suc n) → Expr n
  _∙_ : Expr n → Expr n       → Expr n

Binder : ℕ → Set
Binder = Vec Name

data _⊢_⇝_ : ∀ {n} → Binder n → S.Expr → Expr n → Set where
  var-zero : ∀ {n x} {Γ : Binder n}
           → Γ , x ⊢ var x ⇝ var (# 0)

  var-suc  : ∀ {n x y k} {Γ : Binder n} {p : False (x ≟ y)}
           → Γ ⊢ var x ⇝ var k
           → Γ , y ⊢ var x ⇝ var (suc k)

  lam      : ∀ {n x τ t t′} {Γ : Binder n}
           → Γ , x ⊢ t ⇝ t′
           → Γ ⊢ lam (x ∶ τ) t ⇝ lam τ t′

  _∙_      : ∀ {n t₁ t₁′ t₂ t₂′} {Γ : Binder n}
           → Γ ⊢ t₁ ⇝ t₁′
           → Γ ⊢ t₂ ⇝ t₂′
           → Γ ⊢ t₁ ∙ t₂ ⇝ t₁′ ∙ t₂′
\end{minted}
\end{frame}

\begin{frame}[fragile]
\begin{minted}{cagda}
∅ : Binder 0
∅ = []

Γ : Binder 2
Γ = "x" ∷ "y" ∷ []

e1 : "x" ∷ "y" ∷ [] ⊢ var "x" ⇝ var (# 0)
e1 = var-zero

I : [] ⊢ lam ("x" ∶ A) (var "x")
       ⇝ lam A (var (# 0))
I = lam var-zero

K : [] ⊢ lam ("x" ∶ A) (lam ("y" ∶ A) (var "x"))
       ⇝ lam A (lam A (var (# 1)))
K = lam (lam (var-suc var-zero))

K₂ : [] ⊢ lam ("x" ∶ A) (lam ("y" ∶ A) (var "y"))
        ⇝ lam A (lam A (var (# 0)))
K₂ = lam (lam var-zero)

P : Γ ⊢ lam ("x" ∶ A) (lam ("y" ∶ A) (lam ("z" ∶ A) (var "x")))
      ⇝ lam A (lam A (lam A (var (# 2))))
P = {!!}
\end{minted}
\end{frame}

\section{Typing}
\begin{frame}[fragile]{\texttt{module Typing (U : Set) where}}
\begin{minted}{cagda}
open import Bound Type hiding (_,_)

Ctxt : ℕ → Set
Ctxt = Vec Type

_,_ : ∀ {n} → Ctxt n → Type → Ctxt (suc n)
Γ , x = x ∷ Γ

data _⊢_∶_ : ∀ {n} → Ctxt n → Expr n → Type → Set where
  tVar : ∀ {n Γ} {x : Fin n}
       → Γ ⊢ var x ∶ lookup x Γ

  tLam : ∀ {n} {Γ : Ctxt n} {t} {τ τ′}
       → Γ , τ ⊢ t ∶ τ′
       → Γ ⊢ lam τ t ∶ τ ⟶ τ′

  _∙_  : ∀ {n} {Γ : Ctxt n} {t₁ t₂} {τ τ′}
       → Γ ⊢ t₁ ∶ τ ⟶ τ′
       → Γ ⊢ t₂ ∶ τ
       → Γ ⊢ t₁ ∙ t₂ ∶ τ′
\end{minted}
\end{frame}

% \begin{frame}[allowframebreaks]{References}
% \printbibliography
% \end{frame}


\end{document}