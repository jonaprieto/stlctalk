\documentclass[10pt, xetex, hyperref={pdfpagelabels=false,breaklinks}]{beamer}

% --------------------------------------------------------------------------
% Packages
% --------------------------------------------------------------------------
\usepackage[english]{babel}
\usepackage{amsmath, amsthm, amssymb}
\usepackage{bussproofs}
\EnableBpAbbreviations
\def\extraVskip{3pt}

\usepackage{fancyvrb}
\usepackage{geometry}
\usepackage{graphicx}
\usepackage{lmodern}

\usepackage{url}


% --------------------------------------------------------------------------
% Tikz Configuration
% --------------------------------------------------------------------------
\usepackage{tikz}
\usetikzlibrary{positioning}
\usetikzlibrary{calc}
\usepackage{rotating}

% --------------------------------------------------------------------------
% My palette.
% --------------------------------------------------------------------------
\definecolor{aliceblue}{rgb}{0.94, 0.97, 1.0}
\definecolor{energy}{RGB}{49,247,250}
\definecolor{delicate}{RGB}{67,179,223}
\definecolor{faded}{RGB}{76,117,195}
\definecolor{blu}{RGB}{1,0,102}
\definecolor{plum}{RGB}{87,78,164}
\definecolor{petunias}{RGB}{109,80,139}
\definecolor{letour}{RGB}{101,41,105}
\definecolor{carmine}{rgb}{0.59, 0.0, 0.09}
% --------------------------------------------------------------------------

% --------------------------------------------------------------------------
% Beamer configuration.
% --------------------------------------------------------------------------
\usetheme{default}
\usecolortheme{default}
\usefonttheme{serif}

\beamertemplatenavigationsymbolsempty
\setbeamertemplate{navigation symbols}{}
\hypersetup{pdfpagemode=UseNone}

% footer.
\setbeamercolor{headFoot}{fg=white, bg=blu}
\setbeamertemplate{footline}{
  \leavevmode%
  \hbox{%
  \begin{beamercolorbox}
    [wd=.8\paperwidth,ht=2.3ex,dp=1ex,left]{headFoot}%
    \hspace*{2ex}\textbf\insertshorttitle\hspace*{2mm}|
    \hspace*{2mm}\textbf\insertshortauthor
  \end{beamercolorbox}%
  \begin{beamercolorbox}
    [wd=.2\paperwidth,ht=2.3ex,dp=1ex,right]{headFoot}%
    \insertframenumber{}/\inserttotalframenumber\hspace*{2ex}
  \end{beamercolorbox}}%
  \vskip 0pt%
}

\setbeamerfont{frametitle}{size=\small,series=\bfseries}
\setbeamercolor{frametitle}{fg=white,bg=blu}

\setbeamerfont{framesubtitle}{size=\normalfont\scriptsize}
\setbeamercolor{framesubtitle}{fg=white, bg=blu}

\setbeamercolor{background canvas}{bg=white}
\setbeamercolor{normal text}{fg=black}

% \setbeamercolor{institute}{fg=blu}
\setbeamercolor{title}{fg=blu}
\setbeamercolor{bibliography item}{fg=blu}
% \setbeamercolor{subtitle}{fg=blu}

% \setbeamercolor{titlelike}{fg=blu}
\setbeamerfont{footnote}{size=\tiny}
\setbeamercolor{footnote}{fg=gray}
\setbeamercolor{block title}{bg=white,fg=blu}
% \setbeamercolor{block body}{bg=aliceblue}
\setbeamercolor{item}{fg=blu} % color of bullets
\setbeamercolor{subitem}{fg=blu}
% \setbeamercolor{itemize/enumerate subbody}{fg=blu}
% \setbeamertemplate{itemize subitem}{{\textendash}}
% \setbeamerfont{itemize/enumerate subbody}{size=\footnotesize}
% \setbeamerfont{itemize/enumerate subitem}{size=\footnotesize}

% --------------------------------------------------------------------------
% Fonts
% --------------------------------------------------------------------------
\usefonttheme{professionalfonts}
\usefonttheme{serif}
\usepackage{fontspec}
\usepackage{mathtools}
\usepackage{unicode-math}

\setmonofont[ExternalLocation=fonts/
, BoldFont=DejaVuSansMono-Bold.ttf
, BoldItalicFont=DejaVuSansMono-BoldOblique.ttf
, ItalicFont=DejaVuSansMono-Oblique.ttf
]{DejaVuSansMono.ttf}

\setmathfont[ExternalLocation=fonts/
  , Colour=blu
  ]{DejaVuMathTeXGyre.ttf}
\newfontfamily\mathfont{fonts/DejaVuMathTeXGyre.ttf}

\setmainfont[ExternalLocation=fonts/
  , BoldFont=SourceSansPro-Semibold.otf
  , BoldItalicFont=SourceSansPro-SemiboldIt.otf
  , ItalicFont=SourceSansPro-It.otf
  ]{SourceSansPro-Regular.otf}

% --------------------------------------------------------------------------
% Agda Source code
% --------------------------------------------------------------------------

\usepackage{minted}
\setminted[cagda]{
  bgcolor   = aliceblue
, fontsize  = \footnotesize
, frame     = none
% , framerule = 0.4pt
% , framesep  = 0pt
, style     = cagda
}

% --------------------------------------------------------------------------
% References
% --------------------------------------------------------------------------

\usepackage[autostyle]{csquotes}
\usepackage[
    backend=biber
  , doi=false
  , eprint=false
  , isbn=false
  , natbib=true
  , sortlocale=en_US
  , style=authoryear-icomp
  , url=true
  , block=ragged
]{biblatex}
\addbibresource{ref.bib}
\renewcommand*{\nameyeardelim}{\addcomma\addspace}
\usepackage{silence}
\WarningFilter{biblatex}{Patching footnotes failed}
\WarningFilter{hyperref}{Token not allowed in a PDF string}

% --------------------------------------------------------------------------
% Title and Author
% --------------------------------------------------------------------------

\title[The Simply Typed Lambda Calculus]{The Simply Typed Lambda Calculus}

\subtitle{(In \texttt{Agda})}
\date{\footnotesize 1th June 2017}
\author[Jonathan Prieto-Cubides]{Jonathan Prieto-Cubides}
\institute{
Master in Applied Mathematics\\
Logic and Computation Group\\
Universidad EAFIT\\
Medell\'in, Colombia}
% --------------------------------------------------------------------------

\newsavebox\agdapragma

\begin{document}
\setcounter{page}{1}

\begin{frame}[plain]
\titlepage
  \begin{tikzpicture}[overlay, remember picture]
   \tikzset{shift={(current page.center)}}
    \node[xshift=0cm,yshift=-3.2cm] (eafit)
      {\includegraphics[width=0.2\textwidth]{figures/eafit}};
  \end{tikzpicture}
\end{frame}


\begin{frame}{Contents}
\tableofcontents
\end{frame}

\begin{frame}{About}
\begin{itemize}
\item The Agda source code of this talk is available in the repository
{\color{plum}
\begin{center}
\href{https://github.com/jonaprieto/stlctalk}{https://github.com/jonaprieto/stlctalk}.
\end{center}
}
We present a refactor of the implementation by \citep{cactus} for the simple lambda calculus,
specifically in the Scopecheck and Typecheck module.
\item Tested with Agda v$2.5.2$ and Agda Standard Library v$0.13$
\end{itemize}
\end{frame}

\section{Lambda Calculus}
\begin{frame}[fragile]{Lambda Calculus}
\begin{definition}
\begin{itemize}
\item The set of $\lambda$-terms denoted by $\Lambda$ is built up
from a set of variables $V$ using application and (function) abstraction
\begin{align*}
x\in V                &\Rightarrow x\in \Lambda, \\
M\in \Lambda, x\in V  &\Rightarrow (\lambda x. M) \in \Lambda,\\
M, N\in \Lambda       &\Rightarrow (MN) \in \Lambda.
\end{align*}
\item A simple syntax definition for lambda terms
\vskip 1.5mm
\begin{minted}{cagda}
Name : Set
Name = String

data Expr : Set where
  var : Name → Expr
  lam : Name → Expr → Expr
  _∙_ : Expr → Expr → Expr
\end{minted}
\end{itemize}
\end{definition}
\end{frame}

\section{Typed Lambda Calculus}
\begin{frame}{Lambda Curry System}
\begin{itemize}
\item The set of types is noted with $𝕋 =\text{ Type}(\lambda\rightarrow)$.
$$𝕋 = 𝕍\, |\, 𝔹\, |\, 𝕋 ↣ 𝕋,$$
where $𝕍 = \{α₁,α₂, \cdots \}$ be a set of type variables,
$𝔹$ stands for a collection of type constants for basic types
like \texttt{Nat} or \texttt{Bool}
\item A \emph{statement} is of the form $M : σ$ with $M ∈ Λ$ and $σ ∈ 𝕋$
\item \emph{Derivation} inference rules
\vskip 1mm
\begin{columns}
  \begin{column}{0.5\textwidth}
    \begin{prooftree}
    \AxiomC{$M : σ ↣ τ$}
    \AxiomC{$N : σ$}
    \BinaryInfC{$MN : τ$}
    \end{prooftree}
  \end{column}
  \begin{column}{0.5\textwidth}
    \begin{prooftree}
    \AxiomC{$[x : \sigma]^{(1)}$}
    \UnaryInfC{$\vdots$}
    \UnaryInfC{$M : \tau$}
    \RightLabel{\scriptsize (1)}
    \UnaryInfC{$λx.M : σ ↣ \tau$}
    \end{prooftree}
  \end{column}
\end{columns}
\vskip 1mm
\item A statement $M : \sigma$ is derivable form a \textit{basis} $Γ$ denoted
by $Γ ⊢ M : σ$ where basis stands for be a set of statements with only distinct
(term) variables as subjects
\end{itemize}
\end{frame}

\section{Syntax Definitions}
\begin{frame}[fragile]{Syntax defintion based on ~\citep{cactus}, and \citep{nad}}
\begin{itemize}
  \item Typing syntax: $𝕋 = 𝕍\, |\, 𝔹\, |\, 𝕋 ↣ 𝕋$,
  \vskip 1.5mm
\begin{minted}{cagda}
module Typing (U : Set) where

data Type : Set where
  base : U    → Type
  _↣_  : Type → Type → Type
\end{minted}

\item A syntax definition including type annotations
\vskip 1.5mm
\begin{minted}{cagda}
module Syntax (Type : Set) where

open import Data.String

Name : Set
Name = String

data Formal : Set where
  _∶_ : Name → Type → Formal

data Expr : Set where
  var : Name   → Expr
  lam : Formal → Expr → Expr
  _∙_ : Expr   → Expr → Expr
\end{minted}
\end{itemize}
\end{frame}

\begin{frame}[fragile]{Examples}
\begin{minted}{cagda}
open import Syntax Type

postulate A : Type

x = var "x"
y = var "y"
z = var "z"

-- Combinators.
-- I, K, S : Expr

I = lam ("x" ∶ A) x                  -- λx.x, x : A
K = lam ("x" ∶ A) (lam ("y" ∶ A) x)  -- λxy.x, x,y : A
S =
  lam ("x" ∶ A)
    (lam ("y" ∶ A)
      (lam ("z" ∶ A)
        ((x ∙ z) ∙ (y ∙ z))))        -- λxyz.xz(yz), x,y,z : A
\end{minted}
\end{frame}

\section{Decibility of Type Assignment}
\begin{frame}{Decibility of Type Assignment~\citep{barendregt2013lambda}}

\begin{tabular}{ll}
{\color{blu} \textbf{Problem}} & {\color{blu} \textbf{Question}} \\
{\color{blu} Typability}       & Given $M$ does exists a $σ$ such that $Γ ⊢ M : σ$? \\
{\color{blu} Type-checking}    & Given $M$ and $τ$, can we have $Γ ⊢ M : τ$?  \\
{\color{blu} Inhabitation}     & Given $τ$, does exists an $M$ such that $Γ ⊢ M : σ$?\\
\end{tabular}

\begin{theorem}%[Typability]
\begin{itemize}
\item It is decidable whether a term is typable in $\lambda\rightarrow$.
\item If a term $M$ is typable in $\lambda\rightarrow$, then M has a principal type scheme, i.e.
a type $σ$ such that every possible type for $M$ is a subsitution instance of $σ$.
Moreover $σ$ is computable from $M$.
\end{itemize}
\end{theorem}

\begin{theorem}%[Type-checking]
Type checking for $\lambda\rightarrow$ is decidable.
\end{theorem}

\end{frame}

\section{Well-Scoped Lambda Expressions}
\begin{frame}[fragile]{De Bruijn Index}
\begin{itemize}
\item The indexes are natural numbers that represent the occurrences of the variable in a λ-term
$$  λx. λy. x ⇝  λ λ 2$$
\item The natural number denotes the number of binders that are in scope between that occurrence and its corresponding binder
$$λx. λy. λz. x z (y z)  ⇝ λ λ λ 3 1 (2 1)$$
\item Check for $α$-equivalence is the same as that for syntactic equality
\item A syntax definition using De Bruijn indexes\\[3mm]
\begin{minted}{cagda}
data Expr (n : ℕ) : Set where
  var : Fin n  → Expr n
  lam : Type   → Expr (suc n) → Expr n
  _∙_ : Expr n → Expr n       → Expr n
\end{minted}
\end{itemize}
\end{frame}

\begin{frame}[fragile]{\texttt{module Bound (Type : Set) where}}
% Well-scoped Expresions with respect to a variable Binder.
\begin{minted}{cagda}
Binder : ℕ → Set
Binder = Vec Name

data _⊢_⇝_ : ∀ {n} → Binder n → S.Expr → Expr n → Set where

  var-zero : ∀ {n x} {Γ : Binder n}
           → Γ , x ⊢ var x ⇝ var (# 0)

  var-suc  : ∀ {n x y k} {Γ : Binder n} {p : False (x ≟ y)}
           → Γ ⊢ var x ⇝ var k
           → Γ , y ⊢ var x ⇝ var (suc k)

  lam      : ∀ {n x τ t t′} {Γ : Binder n}
           → Γ , x ⊢ t ⇝ t′
           → Γ ⊢ lam (x ∶ τ) t ⇝ lam τ t′

  _∙_      : ∀ {n t₁ t₁′ t₂ t₂′} {Γ : Binder n}
           → Γ ⊢ t₁ ⇝ t₁′
           → Γ ⊢ t₂ ⇝ t₂′
           → Γ ⊢ t₁ ∙ t₂ ⇝ t₁′ ∙ t₂′
\end{minted}
\end{frame}

\begin{frame}[fragile]{Examples}
\begin{minted}{cagda}
∅ : Binder 0
∅ = []

Γ : Binder 2
Γ = "x" ∷ "y" ∷ []

e1 : "x" ∷ "y" ∷ [] ⊢ var "x" ⇝ var (# 0)
e1 = var-zero

I : [] ⊢ lam ("x" ∶ A) (var "x")
       ⇝ lam A (var (# 0))
I = lam var-zero

K : [] ⊢ lam ("x" ∶ A) (lam ("y" ∶ A) (var "x"))
       ⇝ lam A (lam A (var (# 1)))
K = lam (lam (var-suc var-zero))

K₂ : [] ⊢ lam ("x" ∶ A) (lam ("y" ∶ A) (var "y"))
        ⇝ lam A (lam A (var (# 0)))
K₂ = lam (lam var-zero)

P : Γ ⊢ lam ("x" ∶ A) (lam ("y" ∶ A) (lam ("z" ∶ A) (var "x")))
      ⇝ lam A (lam A (lam A (var (# 2))))
P = {!!}   -- complete!!
\end{minted}
\end{frame}

\begin{frame}[fragile]{\texttt{module Scopecheck (Type : Set) where}}
\begin{minted}{cagda}
name-dec : ∀ {n} {Γ : Binder n} {x y : Name} {t : Expr (suc n)}
         → Γ , y ⊢ var x ⇝ t
         → x ≡ y ⊎ ∃[ t′ ] (Γ ⊢ var x ⇝ t′)

⊢subst : ∀ {n} {x y} {Γ : Binder n} {t}
       → x ≡ y
       → Γ , x ⊢ var x ⇝ t
       → Γ , y ⊢ var x ⇝ t

find-name : ∀ {n}
          → (Γ : Binder n)
          → (x : Name)
          → Dec (∃[ t ] (Γ ⊢ var x ⇝ t))

check : ∀ {n}
      → (Γ : Binder n)
      → (t : S.Expr)
      → Dec (∃[ t′ ] (Γ ⊢ t ⇝ t′))

scope : (t : S.Expr) → {p : True (check [] t)} → Expr 0
scope t {p} = proj₁ (toWitness p)
\end{minted}
\end{frame}

\begin{frame}[fragile]{Examples}
\begin{minted}{cagda}
postulate A : Type

I₁ : S.Expr
I₁ = S.lam ("x" ∶ A) (S.var "x")

open import Data.Unit

I = scope I₁ {p = ⊤.tt}  -- Use C-C-C-n and check for I.

x, y, z : S.Expr
x = var "x"
y = var "y"
z = var "z"

S₁ =
  lam ("x" ∶ A)
    (lam ("y" ∶ A)
      (lam ("z" ∶ A)
        ((x ∙ z) ∙ (y ∙ z))))

S : Expr 0
S = scope S₁ {p = ⊤.tt}  -- Use C-C-C-n and check for S.
\end{minted}
\end{frame}


\begin{frame}{Typing Rules}
\begin{itemize}
\item Introduction\\
\begin{prooftree}
\AxiomC{$Γ(t) = \tau$}
\UnaryInfC{$Γ ⊢ t : \tau$}
\end{prooftree}\hfill

\item Abstraction\\
\begin{prooftree}
\AxiomC{$Γ , τ ⊢ t : σ$}
\UnaryInfC{$Γ ⊢ λ\ τ\ t ∶ τ ↣ \sigma$}
\end{prooftree}\hfill

\item Application
\begin{prooftree}
\AxiomC{$Γ ⊢ t₁ : τ ↣ σ$}
\AxiomC{$Γ ⊢ t₂ : τ$}
\BinaryInfC{$Γ ⊢ t₁ ∙ t₂ ∶ σ$}
\end{prooftree}\hfill
\end{itemize}

\end{frame}

\section{Typability and Type-checking}
\begin{frame}[fragile]{\texttt{module Typing (U : Set) where}}
\begin{minted}{cagda}
open import Bound Type hiding (_,_)

Ctxt : ℕ → Set
Ctxt = Vec Type

_,_ : ∀ {n} → Ctxt n → Type → Ctxt (suc n)
Γ , x = x ∷ Γ

data _⊢_∶_ : ∀ {n} → Ctxt n → Expr n → Type → Set where

  tVar : ∀ {n Γ} {x : Fin n}
       → Γ ⊢ var x ∶ lookup x Γ

  tLam : ∀ {n} {Γ : Ctxt n} {t} {τ σ}
       → Γ , τ ⊢ t ∶ σ
       → Γ ⊢ lam τ t ∶ τ ↣ σ

  _∙_  : ∀ {n} {Γ : Ctxt n} {t₁ t₂} {τ σ}
       → Γ ⊢ t₁ ∶ τ ↣ σ
       → Γ ⊢ t₂ ∶ τ
       → Γ ⊢ t₁ ∙ t₂ ∶ σ
\end{minted}
\end{frame}

\begin{frame}[fragile]{Examples}
\begin{minted}{cagda}
postulate
  Bool : Type

ex : [] , Bool ⊢ var (# 0) ∶ Bool
ex = tVar

ex2 : [] ⊢ lam Bool (var (# 0)) ∶ Bool ↣ Bool
ex2 = tLam tVar

postulate
  Word : Type
  Num  : Type

K : [] ⊢ lam Word (lam Num (var (# 1))) ∶ Word ↣ Num ↣ Word
K = tLam (tLam tVar)
\end{minted}
\end{frame}


\begin{frame}[fragile]{Equality Between Types}
\begin{minted}{cagda}
_T≟_ : (τ τ′ : Type) → Dec (τ ≡ τ′)
base A T≟ base B with A ≟ B
... | yes A≡B = yes (cong base A≡B)
... | no  A≢B = no (A≢B ∘ helper)
  where
    helper : base A ≡ base B → A ≡ B
    helper refl = refl
base A T≟ (_ ↣ _) = no (λ ())

(τ₁ ↣ τ₂) T≟ base B = no (λ ())
(τ₁ ↣ τ₂) T≟ (τ₁′ ↣ τ₂′) with τ₁ T≟ τ₁′
... | no  τ₁≢τ₁′ = no (τ₁≢τ₁′ ∘ helper)
  where
    helper :  τ₁ ↣ τ₂ ≡ τ₁′ ↣ τ₂′ → τ₁ ≡ τ₁′
    helper refl = refl
... | yes τ₁≡τ₁′
  with τ₂ T≟ τ₂′
...  | yes τ₂≡τ₂′ = yes (cong₂ _↣_ τ₁≡τ₁′ τ₂≡τ₂′)
...  | no  τ₂≢τ₂′ = no (τ₂≢τ₂′ ∘ helper)
  where
    helper : τ₁ ↣ τ₂ ≡ τ₁′ ↣ τ₂′ → τ₂ ≡ τ₂′
    helper refl = refl
\end{minted}
\end{frame}
\begin{frame}[fragile]{An Example of an Useful Theorem}
\begin{minted}{cagda}
-- Auxiliar Helper.
⊢-inj : ∀ {n Γ} {t : Expr n} → ∀ {τ σ}
      → Γ ⊢ t ∶ τ
      → Γ ⊢ t ∶ σ
      → τ ≡ σ

-- Var case.
⊢-inj tVar tVar = refl

-- Abstraction case.
⊢-inj {t = lam τ t} (tLam Γ,τ⊢t:τ′) (tLam Γ,τ⊢t:τ″)
  = cong (_↣_ τ) (⊢-inj Γ,τ⊢t:τ′ Γ,τ⊢t:τ″)

-- Application case.
⊢-inj (Γ⊢t₁:τ↣τ₂ ∙ Γ⊢t₂:τ) (Γ⊢t₁:τ₁↣σ ∙ Γ⊢t₂:τ₁)
  = helper (⊢-inj Γ⊢t₁:τ↣τ₂ Γ⊢t₁:τ₁↣σ)
  where
    helper : ∀ {τ τ₂ τ₁ σ} → (τ ↣ τ₂ ≡ τ₁ ↣ σ) → τ₂ ≡ σ
    helper refl = refl
\end{minted}
\end{frame}

\begin{frame}[fragile]{Typability I}
\begin{minted}{cagda}
infer : ∀ {n} Γ (t : Expr n) → Dec (∃[ τ ] (Γ ⊢ t ∶ τ))

-- Var case.
infer Γ (var x) = yes (lookup x Γ -and- tVar)

-- Abstraction case.
infer Γ (lam τ t) with infer (τ ∷ Γ) t
... | yes (σ -and- Γ,τ⊢t:σ) = yes (τ ↣ σ -and- tLam Γ,τ⊢t:σ)
... | no  Γ,τ⊬t:σ = no helper
  where
    helper : ∄[ τ′ ] (Γ ⊢ lam τ t ∶ τ′)
    helper (base A -and- ())
    helper (.τ ↣ σ -and- tLam Γ,τ⊢t:σ)
      = Γ,τ⊬t:σ (σ -and- Γ,τ⊢t:σ)
\end{minted}
\end{frame}

\begin{frame}[fragile]{Typability II}
\begin{minted}{cagda}
-- Application case part I.
infer Γ (t₁ ∙ t₂) with infer Γ t₁ | infer Γ t₂
... | no  ∄τ⟨Γ⊢t₁:τ⟩ | _ = no helper
    where
      helper : ∄[ σ ] (Γ ⊢ t₁ ∙ t₂ ∶ σ)
      helper (τ -and- Γ⊢t₁:τ ∙ _)
        = ∄τ⟨Γ⊢t₁:τ⟩ (_ ↣ τ -and- Γ⊢t₁:τ)

... | yes (base x -and- Γ⊢t₁:base) | _ = no helper
    where
      helper : ∄[ σ ] (Γ ⊢ t₁ ∙ t₂ ∶ σ)
      helper (τ -and- Γ⊢t₁:_↣_ ∙ _)
        with ⊢-inj Γ⊢t₁:_↣_ Γ⊢t₁:base
      ...  | ()
\end{minted}
\end{frame}

\begin{frame}[fragile]{Typability III}
\begin{minted}{cagda}
-- Application case part II.
... | yes (τ₁ ↣ τ₂ -and- Γ⊢t₁:τ₁↣τ₂) | no ∄τ⟨Γ⊢t₂:τ⟩ = no helper
    where
      helper : ∄[ σ ] (Γ ⊢ t₁ ∙ t₂ ∶ σ)
      helper (τ -and- Γ⊢t₁:τ₁′↣τ₂′ ∙ Γ⊢t₂:τ)
        with ⊢-inj Γ⊢t₁:τ₁↣τ₂ Γ⊢t₁:τ₁′↣τ₂′
      ...  | refl = ∄τ⟨Γ⊢t₂:τ⟩ (τ₁ -and- Γ⊢t₂:τ)

... | yes (τ₁ ↣ τ₂ -and- Γ⊢t₁:τ₁↣τ₂) | yes (τ₁′ -and- Γ⊢t₂:τ₁′)
    with τ₁ T≟ τ₁′
...  | yes τ₁≡τ₁′ = yes (τ₂ -and- Γ⊢t₁:τ₁↣τ₂ ∙ helper)
     where
       helper : Γ ⊢ t₂  ∶ τ₁
       helper = subst (_⊢_∶_ Γ t₂) (sym τ₁≡τ₁′) Γ⊢t₂:τ₁′
...  | no  τ₁≢τ₁′ = no helper
     where
       helper : ∄[ σ ] (Γ ⊢ t₁ ∙ t₂ ∶ σ)
       helper (_ -and- Γ⊢t₁:τ↣τ₂ ∙ Γ⊢t₂:τ₁)
         with ⊢-inj  Γ⊢t₁:τ↣τ₂ Γ⊢t₁:τ₁↣τ₂
       ...  | refl = τ₁≢τ₁′ (⊢-inj Γ⊢t₂:τ₁ Γ⊢t₂:τ₁′)
\end{minted}
\end{frame}

\begin{frame}[fragile]{Type-checking I}
\begin{minted}{cagda}
check : ∀ {n} Γ (t : Expr n) → ∀ τ → Dec (Γ ⊢ t ∶ τ)

-- Var case.
check Γ (var x) τ with lookup x Γ T≟ τ
... | yes refl = yes tVar
... | no ¬p    = no (¬p ∘ ⊢-inj tVar)

-- Abstraction case.
check Γ (lam τ t) (base A) = no (λ ())
check Γ (lam τ t) (τ₁ ↣ τ₂) with τ₁ T≟ τ
... | no τ₁≢τ = no (τ₁≢τ ∘ helper)
    where
      helper : Γ ⊢ lam τ t ∶ (τ₁ ↣ τ₂) → τ₁ ≡ τ
      helper (tLam t) = refl

... | yes refl with check (τ ∷ Γ) t τ₂
...               | yes Γ,τ⊢t:τ₂ = yes (tLam Γ,τ⊢t:τ₂)
...               | no  Γ,τ⊬t:τ₂ = no helper
  where
    helper : ¬ Γ ⊢ lam τ t ∶ τ ↣ τ₂
    helper (tLam Γ,τ⊢t:_) = Γ,τ⊬t:τ₂ Γ,τ⊢t:_
\end{minted}
\end{frame}

\begin{frame}[fragile]{Type-checking II}
\begin{minted}{cagda}
-- Application case.
check Γ (t₁ ∙ t₂) σ with infer Γ t₂
... | yes (τ -and- Γ⊢t₂:τ)
    with check Γ t₁ (τ ↣ σ)
...    | yes Γ⊢t₁:τ↣σ = yes (Γ⊢t₁:τ↣σ ∙ Γ⊢t₂:τ)
...    | no  Γ⊬t₁:τ↣σ = no helper
  where
    helper : ¬ Γ ⊢ t₁ ∙ t₂ ∶ σ
    helper (Γ⊢t₁:_↣_ ∙ Γ⊢t₂:τ′)
      with ⊢-inj Γ⊢t₂:τ Γ⊢t₂:τ′
    ...  | refl = Γ⊬t₁:τ↣σ Γ⊢t₁:_↣_

check Γ (t₁ ∙ t₂) σ | no Γ⊬t₂:_ = no helper
  where
    helper : ¬ Γ ⊢ t₁ ∙ t₂ ∶ σ
    helper (_∙_ {τ = σ} t Γ⊢t₂:τ′) = Γ⊬t₂:_ (σ -and- Γ⊢t₂:τ′)
\end{minted}
\end{frame}

\begin{frame}{References}
\printbibliography
\end{frame}


\end{document}